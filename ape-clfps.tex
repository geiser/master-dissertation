
\section{Atividade de introdução: Informação do planejamento instrucional}

O padrão de roteiro ``Atividade de introdução'' é utilizado em atividades de aprendizagem colaborativas com alto grau de interdependência positiva, de modo que a aprendizagem seja atingida com sucesso na atividade colaborativa (Strijbos et al., 2004).

A interdependência positiva é a atitude referida à percepção de como um membro da grupo está vinculado com os outros de forma que ele seja ciente de não ter êxito ao menos que os outros membros tenham exito, ou seja, seu aprendizagem está na aprendizagem de todos os membros grupo (Johnson & Johnson, 1999). A fim de promover a sensação de que os membros da equipe precisam um do outro é necessário deixar informação de como o processo de aprendizagem colaborativa irão realizar-se, de forma que os estudantes entendam: Por que ele vão colaborar? Como vai ser realizada a colaboração (coordenação entre os grupos, etc)? Como é afetado seu desempenho com relação ao desempenho dos outros?

Portanto, deve-se incluir no fluxo de atividades uma atividade que explica o planejamento instrucional com a apresentação da atividade (ou problema) que irão ser resolvida e a apresentação do fluxo (sequência) de atividades que irão ser realizada (incluindo os diferentes grupos que podem se formar), a fim de completar a atividade.

\begin{enumerate}
\item Explicar o planejamento instrucional empregado. As tarefas e/ou problemas precisam ser explicadas de forma clara e mensurável.
\item Explicar os objetivos para garantir a transferência e fixação (objetivos podem ser estabelecidos como resultados).
\item Explicar os princípios e estrategias empregadas no planejamento instrucional e sua relação com os estudantes.
\item Explicar o procedimento (fluxo de atividades, formação de grupos, etc.). Os estudantes são acompanhados nas tarefas completadas.
\item Efetuar perguntas aos estudantes para validar o entendimento do planejamento instrucional.
\item Efetuar perguntas da atividade a ser efetuada que estabelece expectativas e organiza com antecedência o que sabem sobre o tema.
\end{enumerate}

