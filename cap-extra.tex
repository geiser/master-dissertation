

\section{Estudo de caso: Gera\c{c}\~{a}o de cursos no dom\'{i}nio de ci\^{e}ncias da computa\c{c}\~{a}o}

A ferramenta de gera\c{c}\~{a}o de cursos descrita no Capitulo \ref{cap:implementacao} parte do princ\'{i}pio de que o modelo de dom\'{i}nio a ser ensinado pode ser descrito segundo as especifica\c{c}\~{o}es da IMS-GLC (IMS-MD para a descri\c{c}\~{a}o dos objetos de aprendizagem e IMS-RDCEO para a descri\c{c}\~{a}o das compet\^{e}ncias). No entanto, uma das dificuldades no desenvolvimento do presente trabalho foi o fato no uso dessas especifica\c{c}\~{o}es ainda n\~{a}o ser uma realidade. Assim, a etapa de defini\c{c}\~{a}o do dom\'{i}nio a ser ensinado foi realizada manualmente, utilizando-se como base o documento de curricular de ci\^{e}ncias da computa\c{c}\~{a}o do ACM (2008, 2001), da seguinte forma:

\begin{itemize}
\item uso do documento curricular de ci\^{e}ncias da computa\c{c}\~{a}o para efetuar a defini\c{c}\~{a}o de elementos fundamentais na estrutura de elementos de conhecimento. A estrutura consiste de 450 elementos fundamentais (Tabela \ref{tab:});
\item uso do documento curricular de ci\^{e}ncias da computa\c{c}\~{a}o para efetuar a defini\c{c}\~{a}o da estrutura de compet�ncias. A estrutura de compet�ncias consiste de 150 compet�ncias (Tabela \ref{tab:});
\item busca de recursos instrucionais a serem usados como elementos auxiliares na estrutura de elementos de conhecimento em reposit\'{o}rios de objetos de aprendizagem, entre eles: CAREO, Learn-Alberta, Merlot, MIT OpenCourseWare e Cesta (Tabela \ref{tab:}); e
\item busca na Web de recursos instrucionais a serem usados como elementos auxiliares na estrutura de elementos de conhecimento utilizando Google, p\'{a}ginas pessoais de professores, e outros s�tios detalhados na Tabela \ref{tab:}.
\end{itemize}

Com a defini��o das estruturas de elementos de conhecimento e compet�ncias, foi poss�vel efetuar a busca de recursos instrucionais usando os termos extra�dos das estruturas. Como resultado das buscas por recursos instrucionais, foram recuperados 450 recursos (Ap�ndice \ref{ape:}). Na etapa seguinte, cada um dos recursos foi descrito, manualmente por um subconjunto de meta-dados da especifica��o IMS-MD (considerando-se somente aqueles que interessam para os objetivos deste trabalho). As descri��es dos recursos instrucionais foram feitas inspecionando-se o seu conte�do um a um, a fim de identificar cada uma de suas caracter�sticas, por exemplo, n�vel de dificuldade, prerequisitos pedag�gicos, benef�cios pedag�gicos etc. Um exemplo de descri��o de meta-dados para um recurso instrucional de tipo exerc�cio � mostrada na Figura \ref{ref:} .

Uma das dificuldades na defini��o das estruturas foi a identifica��o dos pre-requisitos e benef�cios pedag�gicos dos recursos instrucionais, que exigem uma an�lise minuciosa de cada recurso. Al�m disso, para descrever os meta-dados dos recursos instrucionais foram usados termos da linguagem ingl�s. O objetivo desse procedimento � realizar esse estudo de caso supondo que, no futuro os meta-dados sejam descritos utilizando-se uma grande variedade de termos de outras l�nguas.


\section{Modelo de dom�nio}

Nosso framework de planejamento instrucional apresentado no cap�tulo anterior prov� as principais diretrizes para efetuar a modelagem do dom�nio a ser ensinado. Partindo da premisa de ensino adaptado as carateristicas individuasl de cada estudante, a modelagem de dom�nio � baseada na no��o de compet�ncias como entidade que efetua 
